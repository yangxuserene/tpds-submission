\section{Other Placement Policies}
\label{sec: other_placement_policies}

The fact that hybrid job placement can not eliminate the adversary effect of the ``bully" behavior motivates us to explore other job placement policies. 
\textit{Random Router} placement, which has been studied by Jain et al.\cite{jain-sc14}, is one possible job placement policy for future dragonfly connected systems. 
In the \textit{Random Router} placement policy, a router is identified as ``idle" if all its attached nodes are available. 
Idle routers are randomly picked over the network, and all the compute nodes attached to each router are assigned to a single job exclusively. 
Thus, the allocation made by \textit{Random Router} placement policy can preserve the locality within each router.
Compared with the random placement we discussed in Section~\ref{sec:placement-schemes}, \textit{Random Router} placement makes a trade-off between the extent of randomness and locality. 

\textit{Random Partition} job placement policy is another possible solution for future dragonfly systems.
The partition based job placement has been adopted on torus connected HPC systems such as IBM Blue Gene/P and Blue Gene/Q machines to accommodate their capability computing workload.  
It has been well studied by Zhou and Yang et al.\cite{zhou-ipdps-2015}\cite{Yang-Cluster14}\cite{Yang-ICPADS16}. 
Similarly, a partition could be configured as a portion of a group in a dragonfly network. 
For example, in our simulated dragonfly network, we define a partition as half a group, which consists of four routers. Thus, each partition will consist of 16 compute nodes.
The \textit{Random Partition} job placement policy will assign each job some randomly picked partitions. 
\textit{Random Partition} placement preserves more locality and renders less randomness compared with \textit{Random Router} placement.


\begin{table}[ht]
\begin{center}
\caption{Nomenclature for three different random placement and routing configurations} 
\label{tab: placement routing configs}
\begin{tabular}{l c c c }
\toprule % Top horizontal line
\toprule
&\multicolumn{3}{c}{Routing } \\ 
\cmidrule(l){2-4}
Placement  & Minimal & Adaptive & Prog Adaptive\\ % Column names row
\midrule
%Random  &   RM  &   RA   &  RPA   \\ 
Random &   rand-min  &   rand-adp   &  rand-padp   \\ 
\midrule % In-table horizontal line
%Contiguous  &  CM   &   CA   &  CPA   \\ % Content row 1
RandomRouter &   randR-min  &   randR-adp   &  randR-padp   \\
\midrule
%Random  &   RM  &   RA   &  RPA   \\ 
RandomPartition &   randP-min  &   randP-adp   &  randP-padp   \\ 
\midrule % In-table horizontal line
\bottomrule % Bottom horizontal line
\end{tabular}
\end{center}
\end{table}



\subsection{Individual Application Analysis}
\label{sec:rand3-plcmnt-app-study}

\begin{figure*}[t!]
    \centering
    \begin{subfigure}[t]{0.32\textwidth}
        \centering
        \includegraphics[height=1.3 in]{rand3-plcmt/mg/commtime}
        \caption{MultiGrid Communication Time}
        \label{fig:rand3-plcmt-mg-commtime}
    \end{subfigure}\hfill
    \begin{subfigure}[t]{0.32\textwidth}
        \centering
        \includegraphics[height=1.3 in]{rand3-plcmt/cr/commtime}
        \caption{CrystalRouter Communication Time}
        \label{fig:rand3-plcmt-cr-commtime}
    \end{subfigure}\hfill
    \begin{subfigure}[t]{0.32\textwidth}
        \centering
        \includegraphics[height=1.3 in]{rand3-plcmt/amg/commtime}
        \caption{AMG Communication Time}
        \label{fig:rand3-plcmt-amg-commtime}
    \end{subfigure}
   \caption{Application communication time. Workload~\Rmnum{1} is running with three different random placement policies coupled with three routing configurations.}
   \label{fig:rand3-plcmt-apps-commtime}
\end{figure*}

We present the communication time distribution of each application in Workload~\Rmnum{1} under
three different random placement policies and routing configurations, including the random placement policy we discussed in section~\ref{sec:placement-schemes}. These results are presented in Figure~\ref{fig:rand3-plcmt-apps-commtime}. 

As we discussed in section~\ref{sec:workload-1}, MultiGrid and CrystalRouter prefer random placement for the reason that their traffic can be evenly distributed across the network. 
As shown in Figure~\ref{fig:rand3-plcmt-mg-commtime} and~\ref{fig:rand3-plcmt-cr-commtime}, MultiGrid and CrystalRouter exhibits the similar pattern across three different random placement policies. Both applications suffer performance loss when switching from random (rand) to random router (randR) and random partition(randP).
Random router placement preserves some extent of locality by assigning all the compute nodes attached to a router to MultiGrid, resulting some MPI ranks reside on those adjacent nodes. 
And random partition placement makes it worse by grouping four routers together and preserving more locality.
Local congestion is likely to happen among those adjacent ranks in both applications for the same reason we discussed about contiguous placement in section~\ref{sec:workload-1}. 
When random partition is coupled with minimal routing, communication of both MultiGrid and CrystalRouter suffer more delay due to the congestion on minimal paths. 
Adaptive (progressive) can help alleviate that local congestion by routing packets through both minimal and non-minimal paths adaptively. 

Due to its relative small traffic amount, AMG favors allocation with locality to avoid being ``bullied" by the other applications. 
Random router placement provides allocation with router-level locality. 
However, the four compute nodes attached to each router can not accommodate the 3D nearest neighbor communication pattern of AMG. 
Thus, we can only observe very few performance improvement when AMG switching from random to random router placement in Figure~\ref{fig:rand3-plcmt-amg-commtime}.
When random partition placement is coupled with (progressive) adaptive routing, the performance of AMG is greatly boosted, indicating by the slump in AMG's communication time. 
Random partition placement makes a better trade-off between locality and randomness for AMG by assigning all compute nodes attached to four routers in the same group to AMG. 
Although random partition placement coupled with (progressive) adaptive routing can greatly improve AMG's performance, this doesn't mean that it can reduce the ``bully" effect. 
In fact, the improvement for AMG is achieved by introducing performance degradation to MultiGrid and CrystalRouter. 



\subsection{Key Observations}

Based the experiments presented in section~\ref{sec:rand3-plcmnt-app-study}, we can make the following observations.

\textit{Neither random router nor random partition placement can eliminate the ``bully" effect.}
Both placement policies can not avoid AMG sharing network with MultiGrid and CrystalRouter, which is the root cause of ``bully" effect. Traffic from each application still need to share the local and global channel when traversing from the source to the destination router. 

\textit{MultiGrid and CrystalRouter prefer randomness over locality on dragonfly network.}
Due to their high traffic volume, MultiGrid and CrystalRouter prefer random placement to evenly distribute their traffic. 
Random router placement reduces the randomness by preserving the router-level locality, introducing potential local congestion and performance degradation to both applications. 
Random partition placement makes it worse for MultiGrid and CrystalRouter as partition-level locality has been preserved. 


\textit{AMG benefits from allocation with high extent of locality.} As the least communication intensive application in Workload~\Rmnum{1}, AMG prefers exclusive network resource without sharing with other applications. 
The locality preserved by random router and random partition placement policy can reduce the  sharing of routers between different applications. 
The performance of AMG can be greatly improved when random partition placement is applied.





















